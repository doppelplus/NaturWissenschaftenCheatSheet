% % % % % % % % % % % % % % % % % % % % % % % % % % % % % % % % % % % % % % % %
% LaTeX4EI Template for Cheat Sheets                                Version 1.1
%
% Authors: Emanuel Regnath, Martin Zellner
% Contact: info@latex4ei.de
% Encode: UTF-8, tabwidth = 4, newline = LF
% % % % % % % % % % % % % % % % % % % % % % % % % % % % % % % % % % % % % % % %


% ======================================================================
% Document Settings
% ======================================================================

% possible options: color/nocolor, english/german, threecolumn
% defaults: color, english
\documentclass[eglish/german]{latex4ei/latex4ei_sheet}
\usepackage[ngerman]{babel}
\usepackage{booktabs}
% set document information
\title{Cheat Sheet}
\author{RaoulDuke}						% optional, delete if unchanged
\myemail{@co23:matrix.org}				% optional, delete if unchanged
\mywebsite{www.github.com/doppelplus}	% optional, delete if unchanged


% ======================================================================
% Begin
% ======================================================================
\begin{document}

\IfFileExists{git.id}{\input{git.id}}{}
\ifdefined\GitRevision\mydate{\GitNiceDate\ (git \GitRevision)}\fi

% Title
% ----------------------------------------------------------------------
\maketitle   % requires ./img/Logo.pdf


% Section
% ----------------------------------------------------------------------
\section{Chemische Grundlagen}


\begin{sectionbox}
	\subsection{Formelzeichen}
		\begin{tabular}{ll}
			Dichte & $\rho$\\
			Masse & $m$\\
			molare Masse & $M$\\
			Stoffmenge & $n$\\
			Stoffmengenkonzentration & $c$\\
			Volumen & $V$\\
			Liter	& $l$\\
		\end{tabular}

	\subsection{Dichte}
		$ Dichte =  \frac{Masse}{Volumen}$  $ \rho = \frac{m}{V}$
	
	\subsection{Mol und Molare Masse }
		\textbf {Definition atomare Masseneinheit}\\
		$1u = \frac{1}{12}(\ce{^{12}_{6}C}) = 1,66 \cdot 10^{-24}g$\\
		\textbf {Definition Mol}\\
		1 Mol eines Stoffes sind $6,02 \cdot 10^23$ Teilchen dieses Stoffes.\\
		Im PSDE ist die relative Atomasse gleich der Masse eines Mols in g.\\
		\textbf{Beispiel für Molare Masse eines Moleküls}:\\
		Molare Masse von $H_2O$: $M(H_2O) = 2 \cdot M(H)+ M(O) = 2\cdot 1,0\frac{g}{mol} + 16,0\frac{g}{mol}= 18 \frac{g}{mol})$\\
	
	\subsection{Stoffmenge und Konzentration}
		Stoffmenge: $n = \frac{m}{M}$\\
		Stoffemengenkonzentration : $c = \frac{n}{V}$

\end{sectionbox}

\begin{sectionbox}
	\subsection{Atommodell nach Bohr}
		Hauptschalen entweder 1...8 oder K...R.\\
		Nebenschalen mit maximaler Elektronenanzahl: s(2), p(6), d(10), f(14)\\
		Schalenreihenfolge: $\xrightarrow{\text{1s 2s 2p 3s 3p 4s 3d 4p 5s 4d 5p 6s 4f 5d 6p 7s 5f 6d 7p 8s}}$\\
	
		\subsection{Quantenmechanisches Atommodell}
\end{sectionbox}
		
\begin{sectionbox}
	\subsection{Chemische Bindung}
	\begin{itemize}
		\item Ionenbindung
		\item Elektronenpaarbindung oder kovalente Bindung
		\item metallische Bindung.
	\end{itemize}
	\subsubsection{Ionenbindung}
		Je größer die Differenz der Elektronegativitätswerte der beteiligten Atome ist, desto
		stärker ist der ionische Charakter einer Verbindung ausgeprägt.\\

		Wichtige Anionen:\\
		\begin{tabular}{ll}
			\textbf{Formel} & \textbf{Name}\\
			\midrule
			$SO^{2-}_{4}$ & Sulfat\\
			\midrule
			$SO^{2-}_{3}$ & Sulfit\\
			\midrule
			$HSO^{-}_{4}$ & Hydrogensulfat\\
			\midrule
			$HSO^{-}_{3}$ & Hydrogensulfit\\
			\midrule
			$CO^{2-}_{3}$ & Carbonat\\
			\midrule
			$HCO^{-}_{3}$ & Hydrogencarbonat\\
			\midrule
			$PO^{3-}_{4}$ & Phosphat\\
			\midrule
			$HPO^{2-}_{4}$ & Monohydrogenphosphat\\
			\midrule
			$H_2PO^{2-}_{4}$ & Dihydrogenphosphat\\
			\midrule
			$NO^{-}_{3}$ & Nitrat\\
			\midrule
			$CN^{-}$ & Cyanid.\\
		\end{tabular}
		\\
		Das Verhältnis von Kationen zu Anionen ist immer derart,
		dass das Molekül elektrisch neutral ist.
		
		\subsubsection{Elektronenpaarbindung}
		Kovalenzbindung
	\end{sectionbox}

	\section{Kunststoffe}
		\begin{sectionbox}
			$Polymerisationsgrad = \frac{Molare Masse der Makromolekuele}{Molare Masse der Monomere}$.

			
		\end{sectionbox}

	\section{Korrosion}
	\begin{sectionbox}
		\begin{itemize}
			\item Ausgangsstoff für chemische Reaktion = Edukt.
			\item Resultierende Verbindung aus Reaktion = Produkt.
			\item Gibbs-Helmholtz-Beziehung: $\Delta G = \Delta H - T\Delta S$.
			\begin{itemize}
				\item Wird Energie frei $\Delta G < 0$ exergonischer Vorgang.
				\item Wird Energie verbraucht $\Delta G>0$, endergonischer Vorgang. 
			\end{itemize}
			
			\item Der pH Wert einer Lösung ist der negativ dekadische Logarithmus des Zahlenwertes der Hydroxoniumionenkonzentration.\\
					$pH = -lg\cdot c_{H3O^+}$
		\end{itemize}

		\textbf{Regeln zur Bestimmung der Oxidationszahl}
		\begin{itemize}
			\item Im Element ist die Oxidationszahl immer ±0.
			\item Bei einfachen Ionen entspricht die Oxidationszahl immer der Ionenladung.
			\item Die Summe der Oxidationszahlen aller Atome einer Verbindung ergibt die Gesamtladung der Verbindung.
			\item Fluor besitzt in Verbindungen immer die Oxidationszahl –1.
			\item Sauerstoff besitzt in den meisten Fällen die Oxidationszahl –2.
			\item Wasserstoff besitzt in der Regel die Oxidationszahl +1 (Ausnahme: Hydride).
			\item Metalle besitzen in der Regel positive Oxidationszahlen.
			\item Oxidationszahlen anderer Atome in einer Verbindung werden durch Differenzbildung zur Gesamtladung ermittelt.
			\item Bei kovalenten Verbindungen werden die Elektronenpaare dem elektronegativeren Partner zugeordnet.
		\end{itemize}
		
	\end{sectionbox}

	\section{Physik}
		\begin{sectionbox}
			\subsection{Formelzeichen}
				\begin{tabular}{lll}
					Größe & Formelzeichen & Einheit\\
					Geschwindigkeit & $v$ & $\frac{m}{s}$\\
					Strecke & $s$ & m\\
					Kraft & $F$ & N(Newton)\\
					Fläche & $A$ & m²\\
					Beschleunigung & $a$ & $\frac{m}{s^2}$.\\
					Drehzahl & $n$ & -\\
					Winkelgeschwindigkeit & $\omega$ & 1/s\\
					Frequenz & $f$ & $Hz$\\
					Periodendauer & T & $\frac{1}{f}$\\

				\end{tabular}
			\end{sectionbox}

			\begin{sectionbox}
				\subsection{Bewegungen}
					\textbf{Gleichförmige Bewegung}\\
					$v = \frac{s}{t}$.\\
					Beschleunigung: $a = \frac{\Delta v}{\Delta t}$.\\
					Zurückgelegte s bei gleichmäßiger a:
					$s(t)=s_0 + v_0t+ \frac{a}{2}t^2$\\
					Zurückgelegte s bei bekannter $v_t$ und t:
					$ s = \frac{1}{2}\cdot v \cdot t$\\
					Endgeschwindigkeit bei bekanntem a und s: $\sqrt{2 \cdot a \cdot s}$\\
					\textbf{Kreisförmige Bewegungen}\\
					Umfangsgeschwindigkeit: $v_u = n \cdot 2 \cdot r \cdot \pi$.\\
					Winkelgeschwindigkeit: $\omega = \frac{\Delta\phi}{\Delta t}$\\
					Radialbeschleunigung: $a_{rad}= 4 \cdot \pi^2 \cdot r \cdot n^2$
			\end{sectionbox}

			\begin{sectionbox}
				\subsection{Kräfte}
				2. Newtonscher Bewegungssatz: $F = m \cdot a$\\
				$[F] = [m] \cdot [a] = 1 kg \cdot 1\frac{m}{s^2}= 1\frac{kg \cdot m}{s^2} = 1N$.\\
				Ein Newton ist die Kraft, die eine Masse von 1kg die Beschleunigung von $1m/s^2$ verleiht.\\

				$Drehmonment = Kraft \cdot Hebelarm$\\
				Verhältnis aus Kraft zu Hebelarm: $ \frac{F_1}{F_2} = \frac{l_2}{l_1}$. 

			\end{sectionbox}
% ======================================================================
% End
% ======================================================================
\end{document}
